\begin{mysection}{Heapsort}

\subsection{Heaps}

\vspace{-3mm}
\begin{definition}
A \textit{(binary) heap} data structure is an array object that we can view as a nearly complete binary tree (that is lowest level may be filled up to a point). Each node of the tree corresponds to an element of the array. An array $A$ that represents a heap is an object with an $A.length$ attribute, which counts the number of elements in the array; and an $A.heap-size$ attribute, which represents how many elements in the heap are stored within array $A$; where it is satisfied $0 \leq A.heap-size \leq A.length$.
\end{definition}

%% HEAP TREE AND ARRAY TIKZ FIGURE

\vspace{3mm}
The root of the tree is $A[1]$, and given an index $i$ of a node we can compute the indexes of the parent and left and right child nodes.

\begin{pseudocode}{Parent}{i}
    \RETURN $\floorfunc{i/2}$
\end{pseudocode}
\vspace{2mm}
\begin{pseudocode}{Left}{i}
    \RETURN $2i$
\end{pseudocode}
\vspace{2mm}
\begin{pseudocode}{Right}{i}
    \RETURN $2i + 1$
\end{pseudocode}

\vspace{2mm}
\begin{definition}
There are two kinds of binary heaps: max-heaps and min-heaps. In both kinds, the values in the nodes satisfy a \concept{heap property}. In a \concept{max-heap}, the \concept{max-heap property} is that for every node other than the root, $A[\procedure{Parent}(i)] \geq A[i]$. 
Similarly a \concept{min-heap}, the \concept{min-heap property} is that for every node other than the root, $A[\procedure{Parent}(i)] \leq A[i]$.
\end{definition}

\vspace{3mm}
Viewing the heap as a tree, we have that the \textit{heigth}  of a node in a heap is te number of edges on the longest simple downwad path from the node to a leaf, and the \textit{height of the heap} is the height of its root. Since a heap of $n$ nodes is based on a complete binary tree, we have that its height is $\theta(n)$.


\subsection{Maintaining the heap property}




\end{mysection}
