\begin{mysection}{Fundamental Concepts}

\subsection{What Is a Graph?}
\vspace{-3mm}
\begin{definition}
    A \concept{graph} $G$ is a triple consisting a \concept{vertex set} $V(G)$, an \concept{edge set} $E(G)$, and a relation that associates with each edge two vertices (not necessarily distinct) called its \concept{endpoints}.
\end{definition}

\begin{definition}
A \concept{loop} is an edge whose endpoints are equal. \concept{Multiple edges} are edges having the same pair of endpoints.

A \concept{simple graph} is a graph having no loops or multiple edges. 

    When $u$ and $v$ are endpoints of an edge, they are \concept{adjacent} and are \concept{neighbors}. We write $u \adjacent v$ for ``$u$ is adjacent to $v$''.
\end{definition}

\vspace{3mm}
    We identify a simple graph by its vertex set and edge set, treating the edge set as a set of unordered pairs of vertices ($E \subseteq V \times V$) and writing $e = uv \,(= vu)$ for an edge $e$ with endpoints $u$ and $v$.

The identification of a graph, would be insufficient for multigraphs and digraphs.

\begin{definition}
    The \concept{complement} $\overline{G}$ of a simple graph $G$ is the simple graph with vertex set $V(G)$ defined by $uv \in E\left({\overline{G}}\right)$ if and only if $uv \not\in E(G)$. A \concept{clique} in a graph is a set of pairwise adjacent vertices. An \concept{independet set} in a graph is a set of pairwise nonadjacent vertices.
\end{definition}

\begin{definition}
        A graph $G$ is \concept{bipartite} if $V(G)$ is the union of two disjoint (possibly empty) independent sets called \concept{partites sets} of $G$.
\end{definition}

\begin{definition}
    The \concept{chromatic number} of a graph $G$, written $\chi(G)$, is the minimum number of colors needed to label the vertices so that adjacent vertices receive different colors. A graph $G$ is $k$-\concept{partite} if $V(G)$ can be expressed as the union of $k$ (possibly empty) independent sets.
\end{definition}

\vspace{3mm}
    This generalizes the idea of bipartite graphs, which are $2$-partite. Vertices given the same color must form an independent set, so $\chi(G)$ is the minimum number of independent sets needet to partition $V(G)$.

\begin{definition}
    A \concept{path} is a simple graph whose vertices can be ordered so that two vertices are adjacent if and only if they are consecutive in the list. A \concept{cycle} is a graph with an equal number of vertices and edges whose vertices can be placed around a circle so that two vertices are adjacent if and only if they appear consecutively along the circle.
\end{definition}

\begin{definition}
    A graph $G$ is \concept{connected} if each pair of vertices $G$ belongs to a path; otherwise is \concept{disconnected}.
\end{definition}

\begin{definition}
    A \concept{subgraph} of a graph $G$ is a graph $H$ such that $V(H) \subseteq V(G)$ and $E(H) \subseteq E(G)$, and the assignment of endpoints to edges in $H$ is the same as in $G$. We then can write $H \subseteq G$ and say that ``$G$ contains $H$''.
\end{definition}

\begin{definition}
    Let $G$ be a loopless graph with vertex set $V(G) = \{ v_1, \dotsc, v_n\}$ and edge set $E(G) = \{ e_1, \dotsc, e_m\}$. The \concept{adjacency matrix} of $G$, written $A(G)$, is the $n$-by-$n$ matrix in which entry $a_{i, j}$ is the number of edges in $G$ with endpoints $\{v_i, v_j\}$. The \concept{incidende matrix} $M(G)$ is the $n$-by-$m$ matrix in which entry $m_{i, j}$ is $1$ if $v_i$ is endpoint of $e_j$, otherwise is $0$.
\end{definition}

\begin{definition}
    An \concept{isomorphism} from a simple graph $G$ to a simple graph $H$ is a bijection $f: V(G) \to V(H)$ such that $uv \in E(G)$ if and only if $f(u)f(v) \in E(H)$. We say $G$ is \concept{isomorphic} to $H$, written $G \simeq H$, if there is an isomorphism from $G$ to $H$.
\end{definition}

\begin{proposition}
    The isomorphism relation is an equivalence relation on the set of simple graphs.
\end{proposition}

\vspace{3mm}
When we name a graph  without naming its vertices, we often mean its isomorphism class. Technically ``$H$ is a subgraph of $G$'' means that some subgraph of $G$ is isomorphic to $H$.

\begin{definition}
    The path and cycle with $n$ vertices are denoted $P_n$ and $C_n$, respectively; an \concept{$n$-cycle} is a cycle with $n$ vertices. A \concept{complete graph} is a simple graph whose vertices are pairwise adjacent; the complete graph with $n$ vertices is denoted $K_n$. A \concept{complete bipartite graph} (or \concept{biclique}) is a simple bipartite graph such that two vertices are adjacent if and only if they are in different partite sets; when the sets have sizes $r$ ad $s$, the biclique is denoted $K_{r, s}$.
\end{definition}

\begin{remark}
    The structural properties of a graph are determined by its adjacency relation, thus are preserved under isomorphisms. We can prove that $G$ and $H$ are \textit{not} isomorphic by finding some structural property in which they differ. 

    On the other hand, we can not conclude two graphs are isomorphic by checking only adjacency properties. We must find such isomorphism $f: V(G) \to V(H)$.

    Sometimes we can test isomorphism using the complements. Simple graphs $G$ and $H$ are isomorphic if and only if their complements $\overline{G}$ and $\overline{H}$ are isomorphic.
\end{remark}

\begin{definition}
    A graph is \concept{self-complementary} if it is isomorphic to its complement. A \concept{decomposition} of a graph is a list of subgraphs such that each edge appears in exactly one subgraph in the list.
\end{definition}

\begin{remark}
    An $n$-vertex graph $H$ is self-complementary if and only if $K_n$ has a decomposition consisting of two copies of $H$.
\end{remark}

\begin{definition}
    The \concept{Petersen graph} is the simple graph whose vertices are the $2$-element subsets of a $5$-element set and whose edges are pairs of disjoint $2$-element subsets.
\end{definition}

\begin{proposition}
    If two vertices are nonadjacent in the Petersen graph, then they have exactly one common neighbor.
\end{proposition}

\begin{definition}
    The \concept{girth} of a graph is the length of its shortest cycle. If the graph has no cycle we set it to $\infty$.
\end{definition}








\end{mysection}
