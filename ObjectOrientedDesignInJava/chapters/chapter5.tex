\newpage 

\begin{multicols*}{2}
[\section{Designing Classes and Objects}]



\subsection{The Renter Applet}
    \paragraph{Class Design:} A good design hides much of the complexity of
    a class in the implementation of the class. Ideally, the interface of a
    class should be small and simple.

    There is an exception to the rule: when dealing with very simple classes 
    often there will be more interface than implementation.



\subsection{A System Design Process}
    One approach can be applied to a variety of problems when designing a 
    new program (or system).
    \begin{enumerate}
        \item Determine the requirements of the program.
        \item Identify te classes and objects.
        \item Describe the object collaborations and the classes.
        \item Sketch the user interface.
    \end{enumerate}

    The design process is exploratory and \underline{iterative}. Each step 
    may take the designer forward to the following step or back to a preceding 
    step. If it seems to the designer as though a tentative design is going 
    nowhere, the designer may abandon it and restart the entire process.



\subsection{Determining the Requirements}
    The \underline{first step} of object-oriented design is to determine the 
    general requirements the program must satisfy.

    \tip: Don't try to perfeclty design the system in your head. The trick is 
    to get something down on the paper and then study it and improve it (The 
    simple act of writing will help form ideas).

    \paragraph{Summary Paragraph:} A good way to begin to determine the 
    program requirements is to write a \textit{summary paragraph}. The summary 
    paragraph should view the system from a functional perspective, focusing 
    on the inputs and outputs of the system and on its external. In effect, 
    it should describe the interface of the program rather than the 
    implementation of the program.

    This paragraph must be detailed: specify all but necessary actions.

\subsection{Identifying Classes and Objects}
    The next step in the object-oriented design is to identify al the classes
    and objects that will comprise the program.

    \paragraph{CRC Cards:} An effective way to identify classes and objects 
    is by preparing what are known as CRC (\textit{class-responsiblity-collaboration}) 
    cards. Each CRC card describes a single class in terms of the data 
    attributes, responsibilities, and collaborations of the class. The 
    responsibilities of a class are simply the messages it responds to (that 
    is, the public methods it contains). The collaborations (or, collaborators, 
    if you prefer) of a class are simply other classes that interact with the 
    class by sending it messages or receiving messages from it.

    \tip: CRC cards are prepared from a problem summary statement rather
    than from a completed design. 

    Some suggested ways of finding classes, responsibilities and collaborations
    from the summary paragraph:
    \begin{itemize}
        \item \textbf{Classes:} Read through the problem summary statement 
        and identify nouns and noun phrases, placing them on a list. 
        Often it’s necessary that the design include classes that are merely 
        implicit in the problem summary statement.
        \item \textbf{Responsibilities:} Responsibilities relate to actions. 
        A good starting place for finding responsibilities is in the verbs 
        of the problem summary statement
        \item \textbf{Collaborations:} After you’ve identified the classes 
        and their responsibilities, finding the collaborations is easy. 
        Simply scan the list of responsibilities of each class and identify 
        any other classes it needs in order to fulfill its responsibilities. 
        List these as its collaborations. 
    \end{itemize}


    [...] Design involves a myriad of such decisions, wherein each 
    alternative gains you something, but only at a price. Knowing which to 
    choose is no simple matter. That’s why good designs are usually the 
    result of an iterative process. You make a decision that you later see 
    was more costly than you’d originally thought. A conscientious designer 
    will go back and \underline{try the other alternative}.

    
\subsection{Descibint the Object Collaboration and the Classes}
    \tip: You may also discover that a class for which you prepared a CRC 
    card is not as important as it initially seemed. For any such class, 
    you allocate its responsibilities and attributes to other classes and 
    discard its CRC card.

    \paragraph{Use-Case Scenarios:} Just as CRC cards helped you discover 
    classes, \textit{use-case scenarios} help you discover and describe collaborations. 
    A use-case is simply a transaction or a sequence of related operations 
    that the system performs in response to a user request or event. 

    Walk through each use-case (transaction). During the walk-through, you 
    try to identify the objects involved in the use-case and the messages 
    they exchange. 

    \paragraph{Class Diagrams:} Though CRC cards are helpful during design, 
    they’re not very convenient for later reference. After you’ve completed 
    the collaboration diagrams, you transfer information from each CRC card 
    onto a \textit{class diagram}.


    \begin{center}
    \begin{tikzpicture}
        \umlclass{Renter}{
            -- theLender : Lender      \\ 
            -- INITIAL\_QTY : int = 5  \\
        }{
            + void rentItem(Renter)    \\
            + void returnItem(Renter)  \\
            + void tellResult(String)  \\
        }
    \end{tikzpicture}
    \end{center}

\end{multicols*}
