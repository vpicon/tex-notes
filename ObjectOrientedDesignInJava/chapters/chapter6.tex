\newpage 

\begin{multicols*}{2}
[\section{Round-Trip Design: A Case Study}]

\subsection{Day 1: The System Concept}
    The best way to begin the system design is to develop a problem summary statement (summary paragraph). Offer any suggestions at all of what features and functions the system should have.

    \tip: Begin a brainstorming session, listing functionalities as separated number items.

    Knowing from experience that too much critical analysis can stifle creative problem solving, add as many suggestions to the list, even though you’re convinced they’re irrelevant or worse.

    If no additional ideas have surfaced, review and discuss the functions identified so far. This, may lead to discover additional functions or help understand why one or more of the functions already on the list should be removed.



\subsection{Day 2: Preparing CRC Cards}
    Follow up by making the classes CRC cards, which can be produced from the summary paragraph to identify different classes of the system, ther responsibilities and collaborations.

    \tip: To extract the classes of the system from the summary underline the noun phrases in the problem statement. For each noun phrase, prepare a CRC card. These cards represent potential classes. The list of potential classes is then either expanded or contracted through group discussion. After a set of classes is agreed on, use the back of each card to record the attributes of each class. Don’t be too quick to eliminate classes and attributes, because removing them later is simple; after all, they’re not cast in code yet. Nevertheless, you want to avoid adding classes that are tied to the implementation or technical issues.



\stepcounter{subsection}
\subsection{Day 4: Drawing Collaboration Diagrams} 
    \paragraph{Exploring Use Cases:} When we first wrote up the system definition, you described what you thought the system should do. Now, I’d like you to expand on that idea. We want to divide the system functions into specific operations and then elaborate on each of them. We don’t care about the internal operation of the system just now. What we’re interested in is \underline{how a user would make use of the system}. We call these situations use cases, or scenarios. You might like to think of them as little dramas.
    
    \tip:  The easiest way to do that is by working through use-case scenarios. A scenario is developed by asking, “What happens when the user does this?” Remember, you’re not interested in determining what actually happens at the technical level—that is, which records are deleted or which database tables affected. Instead, you’re interested in describing system interactions from the user’s point of view.


\end{multicols*}
