\newpage 

\begin{multicols*}{2}
[\section{Main Memory}]


\subsection{Background}
    \textit{Memory} consists of a large array of bytes, each with its own address. The CPU fetches instructions from memory according to the value of the program counter. 

    The memory unit sees only a stream of memory addresses; it does not know how they are generated (by the instruction counter, indexing, indirection, literal addresses, and so on) or what they are for (instructions or data). Accordingly, we can ignore how a program generates a memory address. We are interested only in the sequence of memory addresses generated by the running program.

    \subsubsection{Basic Hardware.} Main memory and the registers built into the processor itself are the only general-purpose storage that the CPU can access directly. Therefore, any instructions in execution, and any data being used by the instructions, must be in one of these direct-access storage devices. If the data are not in memory, they must be moved there before the CPU can operate on them.
    

    For proper system operation we must protect the operating system from access by user processes, and processes from accessing other processes' memory. Since the \underline{operating system shouldn’t intervene between the CPU and} \underline{its memory accesses} (because of the resulting performance penalty), this protection is implemented at hardware level.

    First we need to make sure that each process has a separate memory space. We can provide this protection by using two registers: a \textit{base register}, which holds the smallest legal physical memory addres of the process; and the \textit{limit register}, which specifies the size of the range of memory available for the process (so that all memory addresses \texttt{n} that satisfy \texttt{base <= n < base + limit}, are available for a given process with such base and limit registers).


\end{multicols*}
