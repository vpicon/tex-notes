\begin{mysection}{The Basics}

\subsection{Graphs}

\vspace*{-3mm}
\begin{definition}
A \concept{graph} is a pair $G = (V, E)$ of sets such that $E \subseteq V \times V$. The elements of $V$ are the \concept{vertices} (or \concept{nodes}, or \concept{points}) of the graph $G$, the elements of $E$ are its \concept{edges}.

A graph with vertex set $V$ is said to be a graph \concept{on} $V$. The vertex set of a graph $G$ is referred to as $V(G)$, its edge set as $E(G)$.
\end{definition}

\begin{definition}
The number of vertices of a graph $G$ is its \concept{order}, written as $\absolute{G}$; its number of edges is denoted by $\norm{G}$. Graphs are \concept{finite}, \concept{infinite}, \concept{countable} and so, according to their order.
\end{definition}

\begin{definition}
A vertex $v$ is \concept{incident} with an edge $e$ if $v \in e$, then $e$ is an edge at $v$. The two vertices incident with an edge are its \concept{ends}, and an edge \concept{joins} its ends. 

An edge $\{x, y\}$ is usually written as $xy$ (or $yx$). If $x \in X$ and $y \in Y$, then $xy$ is an  \concept{$X\textrm{-}\,Y$ edge}. The set of all $X\textrm{-}\,Y$ edges in a set $E$ is denoted by $E(X, Y)$. The set of all edges in $E$ at a vertex $v$ is denoted by $E(v)$.
\end{definition}

\begin{definition}
Two vertices $x$, $y$ of $G$ are \concept{adjacent}, or \concept{neighbours}, if $xy$ is an edge of $G$. Two edges $e \neq f$ are \concept{adjacent} if they have an end in common. If all the vertices of $G$ are pairwise adjacent, then $G$ is \concept{complete}. A complete graph on $n$ vertices is a $K^n$; a $K^3$ is called a \concept{triangle}.
\end{definition}

\begin{definition}
Let $G = (V, E)$ and $G' = (V', E')$ be two graphs. We say $G$ and $G'$ are \concept{isomorphic}, and write $G \simeq G'$, if there exists a bijection $\varphi: V \to V'$ with $xy \in E \Leftrightarrow \varphi(x)\varphi(y) \in E'$. Such a map $\varphi$ is called an \concept{isomorphism}. 
\end{definition}

\vspace{3mm}
We do not distinguish between isomorphic graphs, and write $G = G'$ instead of $G \simeq G'$, and talk about \textit{the} graph.

\begin{definition}
A class of graphs that is closed under isomorphism is called a \concept{graph property}. 
A map taking graphs as arguments is called a \concept{graph invariant} if it assigns equal values to isomorphic graphs. 
\end{definition}

\vspace{3mm}
For example, ``containing a triangle'' is a graph property: if $G$ contains three pairwise adjacent vertices then so does every graph isomorphic to $G$. 

The number of vertices and the number of edges of a graph are two simple graph invariants; the greatest number of pairwise adjacent vertices is another.

\begin{definition}
We set $G \cup G' := (V \cup V', E \cup E')$, and $G \cap G' := (V \cup V', E \cap E')$. If $G \cap G' = \emptyset$, then $G$ and $G'$ are \concept{disjoint}. If $V' \subseteq V$ and $E' \subseteq E$, then $G'$ is a \concept{subgraph} of $G$ (and $G$ a \concept{supergraph} of $G'$), written as $G' \subseteq G$.

If $G' \subseteq G$ and $G'$ contains all the edges $xy \in E$ with $x, y \in V'$, then $G'$ is an \concept{induced subgraph} of $G$; we say that $V'$ \concept{induces} or \concept{spans} $G'$ in $G$ and write $G' =: G\left [V'\right ]$. If $U \subseteq V$ is any set of vertices, then $G\left [U \right ]$ denotes the graph on $U$ whose edges are precisely the edges of $G$ with both ends in $U$. If $H$ is a subgraph of G, not necessarily induced, we write $G\left[H\right] \equiv G\left[V(H)\right]$. Finally, $G' \subseteq G$ is a \concept{spanning subgraph} of $G$ is $V'$ spans all of $G$, i.e. if $V' = V$.

If $U$ is any set of vertices (usually of $G$), we write $G - U$ for $G\left[V \setminus U\right]$. In other words, $G - U$ is obtained from $G$ by deleting all the vertices in $U \cap V$ and their incident edges. We write $G - G'$ instead of $G - V(G')$. For $F \subseteq V \times V$ we write $G - F := (V, E \setminus F)$ and $G + F := (V, E \cup F)$. We call $G$ \concept{edge-maximal} with a given graph property if $G$ itself has the property but no graph $G + xy$ does, for non-adjacent vertices $x, y \in G$.
\end{definition}

\vspace{3mm} More generally, when we call a graph \concept{minimal} or \concept{maximal} with some property but have not specified any particular ordering, we are referring to the subgraph relation. When we speak of minimal or maximal sets of vertices or edges, the reference is simply to set inclusion.

\begin{definition}
If $G$ and $G'$ are disjoint, we denote by $G \ast G'$ the graph obtained from $G \cup G'$ by joining all vertices of $G$ to all vertices of $G'$. The \concept{complement} $\overline{G}$ of $G$ is the graph on $V$ with edge set $V \times V \, \setminus E$. The \concept{line graph} $L(G)$ of $G$ is the graph on $E$ in which $x, y \in E$ are adjacent vertices if and only if they are adjacent edges in $G$.
\end{definition}



\subsection{The degree of a vertex}

\vspace{-3mm}
\begin{definition}
Let $G = (V, E)$ a (non-empty) graph. The set of neighbours of a vertex $v \in G$ is denoted by $N_G (v)$ (or briefly $N(v)$). More generally, for $U \subseteq V$, the neighbours in $V \setminus U$ of vertices of $U$ are called the \concept{neighbours of $U$}, denoted $N(U)$.
\end{definition}

\begin{definition}
The \concept{degree} of a vertex $v \in G$, $d_G(v)$ (or $d(v)$ for short), is the number $\absolute{E(v)}$ of edges at $v$; which by definitino equals the number of neighbors of $v$, $\absolute{N_G(v)}$.

A vertex of degree $0$ is \concept{isolated}. The number $\delta(G) := \min\{d(v) | v \in V \}$ is the \concept{minimum degree} of $G$, the number $\Delta(G) := \max\{d(v) | v \in V \}$ is the \concept{maximum degree} of $G$. 

If all the vertices of $G$ have the same degree $k$, then $G$ is \concept{$k$-regular}.

The number $d(G) := \frac{1}{\absolute{V}} \sum_{v \in V} d_G(v)$, is the \concept{average degree} of $G$.
\end{definition}

\vspace{3mm}
Clearly we have that $\delta(G) \leq d(G) \leq \Delta(G)$. The average degree quantifies globally what is measured locally by the vertex degrees: the number of edges of $G$ per vertex. Sometimes it will be convenient to express this ratio directly, as $\varepsilon(G) := \absolute{E}/\absolute{V}$. Which relates to the average degree:
\[
\absolute{E} = \frac{1}{2}\sum_{v \in G}d(v) = \frac{1}{2}d(G) \cdot \absolute{V} \implies \varepsilon(G) = \frac{1}{2}d(G).
\]

\begin{proposition}
The number of vertices of odd degree in a graph is always even.
\end{proposition}



\end{mysection}
