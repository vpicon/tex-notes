\begin{mysection}{The Relational Model of Data}

\subsection{Data Models}
A \concept{data model} is a notation for describing data or information, which consists in three parts.

\begin{enumerate}[label=\roman*)]
    \item \textit{Structure of data.} The data structures used to implement (in high level languages i.e. C, Java $\dots$) data in the computer are sometimes referred to as \concept{physical data model}. In the database world, data models are higher level than data structures, thus called \concept{conceptual model}.
    \item \textit{Operations on the data.} Generally allowed to perform a \underline{limited} set of operations: \concept{queries} and \concept{modifications}. This limitation makes possible to describe database operations at a very high level, yet the DBMS implement the operations efficiently.
    \item \textit{Constraints on the data.} 
\end{enumerate}


We can differentiate three families of data models for database systems.

\begin{enumerate}[label=\roman*)]
    \item \textit{Relational Model.} Based on tables, called \concept{relations}. Operations associated are table-oriented and form the ``relational algebra''. Mostly used with SQL language.
    \item \textit{Semistructured Model.} Resemble trees or graphs, rather than tables or arrays. Operations involve following paths from an element to its nested subelements, then to subelements nested within those. May use XML or JSON as a language to represent data.
    \item \textit{Other Data Models.} Among those we find a trend to add object-oriented features to the relational model. Values can have structure, rather than being elementary types (integers, strings ...), and relations can have associated methods. Called \concept{object-relational} model. Other fallen models include the \concept{hierarchical model} and the \concept{network model} which was a graph-oriented, physical level model.
\end{enumerate}

\subsubsection{Comparison of Modeling Approaches}

\subsection{Basics of the Relational Model}
\begin{definition}
The \concept{relational model} gives us a single way to represent data: as a two-dimensional table called \concept{relation}, where each row represents an \concept{entry} or \concept{tuple} in the table, and the columns, called \concept{attributes}, represent a property of the entries.
\end{definition}

As an example consider the following relation called \texttt{Movies}.
\vspace{-2mm}
\begin{center}
\begin{tabular}{ >{\ttfamily}l | >{\ttfamily}l | >{\ttfamily}l | >{\ttfamily}l }
\textit{\textrm{title}}   & \textit{\textrm{year}} & \textit{\textrm{length}}& \textit{\textrm{genre}}  \\ 
\hline \hline
Gone With the Wind & 1939 & 231    & drama  \\
Star Wars          & 1977 & 124    & sciFi  \\
Wayne's World      & 1992 & 95     & comedy \\
\end{tabular}
\end{center}

\begin{definition}
The name of a relation and the set of attributes for a relation is called the \concept{schema} for that relation. We can notate a schema with the name and a parenthesized list of attributes.

In the relational model, a database consists of one or more relations. The set of schemas for the relatios of a database forms the \concept{relational database schema}.
\end{definition}

For example for the \texttt{Movies} relation we can write the schema as \texttt{Movies(title, year, length, genre)}. \\


The relational model requires that each component of each tuple be atomic (some elementary type such as integer or string). It is not permitted for a value to be a compound type (structure, set, list, array ...).  
Associated with each attribute of a relation is a \concept{domain}, a particular elementary type. The components of any tuple of the relation must have, in each component, a value that belongs to the domain of the corresponding column. 

We can include the domain in the scheme notation. For example for the \texttt{Movies} relation, \texttt{Movies(title:string, year:integer, length:integer, genre:string)}.

\paragraph{Relation Instances.} A relation is no static: it changes over time. We can insert tuples, delete, and modify them. It is less common, though possible, for the schema of a relaition to change.

\begin{definition}
    A set of tubles for a given relation is an \concept{instance} of that relation. The instance that a DBS maintains in present time is called the \concept{current instance}.
\end{definition}





\end{mysection}
